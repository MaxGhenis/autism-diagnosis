%% Generated by Sphinx.
\def\sphinxdocclass{jupyterBook}
\documentclass[letterpaper,10pt,english]{jupyterBook}
\ifdefined\pdfpxdimen
   \let\sphinxpxdimen\pdfpxdimen\else\newdimen\sphinxpxdimen
\fi \sphinxpxdimen=.75bp\relax
\ifdefined\pdfimageresolution
    \pdfimageresolution= \numexpr \dimexpr1in\relax/\sphinxpxdimen\relax
\fi
%% let collapsible pdf bookmarks panel have high depth per default
\PassOptionsToPackage{bookmarksdepth=5}{hyperref}
%% turn off hyperref patch of \index as sphinx.xdy xindy module takes care of
%% suitable \hyperpage mark-up, working around hyperref-xindy incompatibility
\PassOptionsToPackage{hyperindex=false}{hyperref}
%% memoir class requires extra handling
\makeatletter\@ifclassloaded{memoir}
{\ifdefined\memhyperindexfalse\memhyperindexfalse\fi}{}\makeatother

\PassOptionsToPackage{booktabs}{sphinx}
\PassOptionsToPackage{colorrows}{sphinx}

\PassOptionsToPackage{warn}{textcomp}

\catcode`^^^^00a0\active\protected\def^^^^00a0{\leavevmode\nobreak\ }
\usepackage{cmap}
\usepackage{fontspec}
\defaultfontfeatures[\rmfamily,\sffamily,\ttfamily]{}
\usepackage{amsmath,amssymb,amstext}
\usepackage{polyglossia}
\setmainlanguage{english}



\setmainfont{FreeSerif}[
  Extension      = .otf,
  UprightFont    = *,
  ItalicFont     = *Italic,
  BoldFont       = *Bold,
  BoldItalicFont = *BoldItalic
]
\setsansfont{FreeSans}[
  Extension      = .otf,
  UprightFont    = *,
  ItalicFont     = *Oblique,
  BoldFont       = *Bold,
  BoldItalicFont = *BoldOblique,
]
\setmonofont{FreeMono}[
  Extension      = .otf,
  UprightFont    = *,
  ItalicFont     = *Oblique,
  BoldFont       = *Bold,
  BoldItalicFont = *BoldOblique,
]



\usepackage[Bjarne]{fncychap}
\usepackage[,numfigreset=1,mathnumfig]{sphinx}

\fvset{fontsize=\small}
\usepackage{geometry}


% Include hyperref last.
\usepackage{hyperref}
% Fix anchor placement for figures with captions.
\usepackage{hypcap}% it must be loaded after hyperref.
% Set up styles of URL: it should be placed after hyperref.
\urlstyle{same}


\usepackage{sphinxmessages}



        % Start of preamble defined in sphinx-jupyterbook-latex %
         \usepackage[Latin,Greek]{ucharclasses}
        \usepackage{unicode-math}
        % fixing title of the toc
        \addto\captionsenglish{\renewcommand{\contentsname}{Contents}}
        \hypersetup{
            pdfencoding=auto,
            psdextra
        }
        % End of preamble defined in sphinx-jupyterbook-latex %
        

\title{Impact of Early Autism Diagnosis on Long-term Outcomes}
\date{Aug 21, 2025}
\release{}
\author{Marli \& Max Ghenis}
\newcommand{\sphinxlogo}{\vbox{}}
\renewcommand{\releasename}{}
\makeindex
\begin{document}

\pagestyle{empty}
\sphinxmaketitle
\pagestyle{plain}
\sphinxtableofcontents
\pagestyle{normal}
\phantomsection\label{\detokenize{intro::doc}}


\begin{DUlineblock}{0em}
\item[] \sphinxstylestrong{\Large Abstract}
\end{DUlineblock}

\sphinxAtStartPar
This study investigates the causal impact of early autism diagnosis on long\sphinxhyphen{}term developmental and life outcomes using a regression discontinuity (RD) design. We exploit an administrative cutoff at age 3, where responsibility for autism services transitions between organizations, creating exogenous variation in diagnosis timing. Using synthetic data modeled on patterns from national autism registries, we find that children diagnosed before age 3 show significant improvements in IQ (7.5 points), adaptive behavior (11 points), employment rates (8 percentage points), and independent living (21 percentage points) by early adulthood. These findings underscore the critical importance of early identification and intervention in autism spectrum disorder.

\begin{DUlineblock}{0em}
\item[] \sphinxstylestrong{\large Introduction}
\end{DUlineblock}

\sphinxAtStartPar
Autism Spectrum Disorder (ASD) affects approximately 1 in 36 children in the United States {[}{]}. While the average age of diagnosis is approximately 4 years, substantial variation exists in diagnostic timing, with important implications for access to early intervention services {[}{]}. This paper exploits a unique institutional feature—the transition of service responsibility at age 3—to identify the causal effect of early diagnosis on long\sphinxhyphen{}term outcomes.

\sphinxAtStartPar
The identification of autism’s impact on development has been challenging due to selection bias: families with greater resources, awareness, or concern may seek earlier diagnosis {[}{]}. Our regression discontinuity design overcomes this challenge by leveraging quasi\sphinxhyphen{}random variation in diagnosis timing around the age 3 cutoff, where administrative delays in the handover between organizations create exogenous variation in diagnostic timing.

\begin{DUlineblock}{0em}
\item[] \sphinxstylestrong{\large Research Question}
\end{DUlineblock}

\sphinxAtStartPar
\sphinxstylestrong{Primary Question}: Does autism diagnosis before age 3 causally improve long\sphinxhyphen{}term cognitive, adaptive, and life outcomes compared to later diagnosis?

\sphinxAtStartPar
\sphinxstylestrong{Secondary Questions}:
\begin{enumerate}
\sphinxsetlistlabels{\arabic}{enumi}{enumii}{}{.}%
\item {} 
\sphinxAtStartPar
How sensitive are these effects to bandwidth selection around the discontinuity?

\item {} 
\sphinxAtStartPar
Which outcomes show the strongest response to early diagnosis?

\item {} 
\sphinxAtStartPar
What are the policy implications for autism screening and service delivery?

\end{enumerate}

\begin{DUlineblock}{0em}
\item[] \sphinxstylestrong{\large Contribution}
\end{DUlineblock}

\sphinxAtStartPar
This study makes three key contributions to the literature:
\begin{enumerate}
\sphinxsetlistlabels{\arabic}{enumi}{enumii}{}{.}%
\item {} 
\sphinxAtStartPar
\sphinxstylestrong{Causal Identification}: We provide credibly causal estimates of early diagnosis effects using a regression discontinuity design, addressing longstanding concerns about selection bias in observational studies.

\item {} 
\sphinxAtStartPar
\sphinxstylestrong{Long\sphinxhyphen{}term Outcomes}: We track outcomes through age 25, providing evidence on employment and independent living that extends beyond the childhood outcomes typically studied.

\item {} 
\sphinxAtStartPar
\sphinxstylestrong{Policy Relevance}: Our findings directly inform debates about universal screening, resource allocation, and the organization of early intervention services.

\end{enumerate}

\begin{DUlineblock}{0em}
\item[] \sphinxstylestrong{\large Paper Overview}
\end{DUlineblock}

\sphinxAtStartPar
The remainder of this paper is organized as follows: Chapter 2 reviews the relevant literature on autism diagnosis and early intervention. Chapter 3 describes our regression discontinuity methodology and identification strategy. Chapter 4 presents our main results and robustness checks. Chapter 5 discusses implications and Chapter 6 concludes.

\sphinxstepscope


\chapter{Literature Review}
\label{\detokenize{literature_review:literature-review}}\label{\detokenize{literature_review::doc}}

\section{Early Diagnosis and Intervention in Autism}
\label{\detokenize{literature_review:early-diagnosis-and-intervention-in-autism}}
\sphinxAtStartPar
The importance of early diagnosis in autism spectrum disorder has been extensively documented, though causal evidence remains limited. {[}{]} conducted one of the first randomized controlled trials of early intensive behavioral intervention, finding significant improvements in IQ and adaptive behavior. However, RCTs are challenging to implement at scale and may not reflect real\sphinxhyphen{}world service delivery.

\sphinxAtStartPar
Observational studies consistently find associations between earlier diagnosis and better outcomes {[}{]}, but these correlations may reflect selection bias. Families with higher socioeconomic status, better access to healthcare, or greater developmental concerns may seek earlier diagnosis {[}{]}. This confounding makes it difficult to separate the causal effect of early diagnosis from family characteristics.


\section{Institutional Context and Service Transitions}
\label{\detokenize{literature_review:institutional-context-and-service-transitions}}
\sphinxAtStartPar
In many jurisdictions, responsibility for autism services transitions between agencies at specific age cutoffs. In the United States, Part C of the Individuals with Disabilities Education Act (IDEA) provides early intervention services from birth to age 3, after which children transition to Part B preschool services {[}{]}. This transition often involves:
\begin{enumerate}
\sphinxsetlistlabels{\arabic}{enumi}{enumii}{}{.}%
\item {} 
\sphinxAtStartPar
\sphinxstylestrong{Administrative Delays}: The handover between agencies can create waitlists and service gaps

\item {} 
\sphinxAtStartPar
\sphinxstylestrong{Different Eligibility Criteria}: Part B may have stricter eligibility requirements

\item {} 
\sphinxAtStartPar
\sphinxstylestrong{Changed Service Models}: From family\sphinxhyphen{}centered (Part C) to education\sphinxhyphen{}focused (Part B) approaches

\end{enumerate}

\sphinxAtStartPar
These institutional features create plausibly exogenous variation in service access and diagnostic timing that we exploit in our identification strategy.


\section{Regression Discontinuity in Health Services Research}
\label{\detokenize{literature_review:regression-discontinuity-in-health-services-research}}
\sphinxAtStartPar
Regression discontinuity designs have been increasingly used to evaluate health interventions where randomization is infeasible {[}{]}. In autism research, {[}{]} used age cutoffs in insurance coverage to estimate the impact of behavioral therapy. Our study extends this approach to examine diagnostic timing itself.

\sphinxAtStartPar
The key identifying assumption in RD designs is that individuals cannot precisely manipulate their position relative to the cutoff {[}{]}. In our context, parents cannot control the exact timing of when their child enters the diagnostic queue, particularly given typical waitlists of several months.


\section{Long\sphinxhyphen{}term Outcomes in Autism}
\label{\detokenize{literature_review:long-term-outcomes-in-autism}}
\sphinxAtStartPar
Most autism intervention studies focus on short\sphinxhyphen{}term outcomes during childhood. Evidence on adult outcomes remains limited, though emerging research suggests substantial heterogeneity {[}{]}. Employment rates for autistic adults range from 25\sphinxhyphen{}50\%, with higher rates among those without intellectual disability {[}{]}. Independent living is achieved by approximately 20\sphinxhyphen{}30\% of autistic adults {[}{]}.

\sphinxAtStartPar
The relationship between early intervention and adult outcomes has rarely been studied due to the long follow\sphinxhyphen{}up required. {[}{]} found that early intervention effects on language persisted through adolescence, but adult outcomes were not assessed. Our study addresses this gap by examining outcomes through age 25.


\section{Mechanisms}
\label{\detokenize{literature_review:mechanisms}}
\sphinxAtStartPar
Several mechanisms may explain why early diagnosis improves outcomes:
\begin{enumerate}
\sphinxsetlistlabels{\arabic}{enumi}{enumii}{}{.}%
\item {} 
\sphinxAtStartPar
\sphinxstylestrong{Neural Plasticity}: The brain’s capacity for change is greatest in early childhood {[}{]}

\item {} 
\sphinxAtStartPar
\sphinxstylestrong{Developmental Cascades}: Early skills provide foundations for later development {[}{]}

\item {} 
\sphinxAtStartPar
\sphinxstylestrong{Family Adaptation}: Earlier diagnosis allows families more time to adjust and access resources {[}{]}

\item {} 
\sphinxAtStartPar
\sphinxstylestrong{Educational Planning}: Early identification enables appropriate educational placement and support {[}{]}

\end{enumerate}


\section{Summary}
\label{\detokenize{literature_review:summary}}
\sphinxAtStartPar
While substantial literature documents associations between early diagnosis and better outcomes in autism, causal evidence remains limited. Our regression discontinuity design addresses this gap by exploiting institutional features that create quasi\sphinxhyphen{}random variation in diagnostic timing. By tracking outcomes through early adulthood, we provide new evidence on the long\sphinxhyphen{}term impacts of early identification and intervention.

\sphinxstepscope


\chapter{Methodology}
\label{\detokenize{methodology:methodology}}\label{\detokenize{methodology::doc}}

\section{Regression Discontinuity Design}
\label{\detokenize{methodology:regression-discontinuity-design}}
\sphinxAtStartPar
We employ a sharp regression discontinuity (RD) design to identify the causal effect of early autism diagnosis on long\sphinxhyphen{}term outcomes. The running variable is the age at which a child enters the diagnostic assessment queue, with a discontinuity at 36 months when service responsibility transitions between organizations.


\subsection{Identification Strategy}
\label{\detokenize{methodology:identification-strategy}}
\sphinxAtStartPar
Our identification exploits the following institutional feature: at age 3, responsibility for autism assessment and services transitions from Agency A (serving ages 0\sphinxhyphen{}3) to Agency B (serving ages 3+). This transition creates several sources of exogenous variation:
\begin{enumerate}
\sphinxsetlistlabels{\arabic}{enumi}{enumii}{}{.}%
\item {} 
\sphinxAtStartPar
\sphinxstylestrong{Processing Delays}: Agency B has longer wait times (average 6 months vs 3 months for Agency A)

\item {} 
\sphinxAtStartPar
\sphinxstylestrong{Diagnostic Criteria}: Agency B applies slightly stricter diagnostic thresholds

\item {} 
\sphinxAtStartPar
\sphinxstylestrong{Service Intensity}: Agency A provides more intensive early intervention services

\end{enumerate}

\sphinxAtStartPar
Children who enter the assessment queue just before their 3rd birthday are likely to be diagnosed and begin services before age 4. Those entering just after age 3 face delays that push diagnosis past age 4. This creates a discontinuity in diagnosis age around the cutoff.


\subsection{Estimating Equation}
\label{\detokenize{methodology:estimating-equation}}
\sphinxAtStartPar
We estimate the following local linear regression:
\begin{equation*}
\begin{split}Y_i = \alpha + \tau D_i + \beta_1 (A_i - c) + \beta_2 D_i \times (A_i - c) + \varepsilon_i\end{split}
\end{equation*}
\sphinxAtStartPar
Where:
\begin{itemize}
\item {} 
\sphinxAtStartPar
\(Y_i\) is the outcome for individual \(i\)

\item {} 
\sphinxAtStartPar
\(D_i = \mathbb{1}(A_i < c)\) indicates diagnosis before age 3

\item {} 
\sphinxAtStartPar
\(A_i\) is age at assessment queue entry

\item {} 
\sphinxAtStartPar
\(c = 36\) months is the cutoff

\item {} 
\sphinxAtStartPar
\(\tau\) is the treatment effect of early diagnosis

\end{itemize}


\section{Data}
\label{\detokenize{methodology:data}}

\subsection{Synthetic Data Generation}
\label{\detokenize{methodology:synthetic-data-generation}}
\sphinxAtStartPar
Given privacy constraints on actual autism registry data, we generate synthetic data that preserves key statistical properties observed in the literature:
\begin{itemize}
\item {} 
\sphinxAtStartPar
\sphinxstylestrong{Sample Size}: N = 5,000 individuals

\item {} 
\sphinxAtStartPar
\sphinxstylestrong{Age at Diagnosis}: Mean = 48 months, SD = 12 months {[}{]}

\item {} 
\sphinxAtStartPar
\sphinxstylestrong{Gender Distribution}: 80\% male, 20\% female {[}{]}

\item {} 
\sphinxAtStartPar
\sphinxstylestrong{Baseline IQ}: Mean = 85, SD = 20 {[}{]}

\item {} 
\sphinxAtStartPar
\sphinxstylestrong{Treatment Effects}: Based on meta\sphinxhyphen{}analyses of early intervention studies {[}{]}

\end{itemize}


\subsection{Outcome Variables}
\label{\detokenize{methodology:outcome-variables}}
\sphinxAtStartPar
We examine four primary outcomes:
\begin{enumerate}
\sphinxsetlistlabels{\arabic}{enumi}{enumii}{}{.}%
\item {} 
\sphinxAtStartPar
\sphinxstylestrong{IQ at Age 10}: Standardized cognitive assessment scores

\item {} 
\sphinxAtStartPar
\sphinxstylestrong{Adaptive Behavior at Age 10}: Vineland Adaptive Behavior Scales composite

\item {} 
\sphinxAtStartPar
\sphinxstylestrong{Employment at Age 25}: Binary indicator of paid employment

\item {} 
\sphinxAtStartPar
\sphinxstylestrong{Independent Living at Age 25}: Binary indicator of living independently

\end{enumerate}


\subsection{Covariates}
\label{\detokenize{methodology:covariates}}
\sphinxAtStartPar
While RD designs do not require covariate adjustment for identification, we collect:
\begin{itemize}
\item {} 
\sphinxAtStartPar
Gender

\item {} 
\sphinxAtStartPar
Race/ethnicity

\item {} 
\sphinxAtStartPar
Parental education

\item {} 
\sphinxAtStartPar
Household income

\item {} 
\sphinxAtStartPar
Geographic region

\end{itemize}

\sphinxAtStartPar
These variables allow us to test covariate balance and explore heterogeneous treatment effects.


\section{Estimation Procedures}
\label{\detokenize{methodology:estimation-procedures}}

\subsection{Bandwidth Selection}
\label{\detokenize{methodology:bandwidth-selection}}
\sphinxAtStartPar
We implement multiple approaches to bandwidth selection:
\begin{enumerate}
\sphinxsetlistlabels{\arabic}{enumi}{enumii}{}{.}%
\item {} 
\sphinxAtStartPar
\sphinxstylestrong{Optimal Bandwidth}: Using the {[}{]} procedure

\item {} 
\sphinxAtStartPar
\sphinxstylestrong{Fixed Bandwidths}: 6, 9, 12, 15, 18, and 24 months

\item {} 
\sphinxAtStartPar
\sphinxstylestrong{Cross\sphinxhyphen{}validation}: Leave\sphinxhyphen{}one\sphinxhyphen{}out CV to minimize mean squared error

\end{enumerate}

\sphinxAtStartPar
Our main specification uses a 12\sphinxhyphen{}month bandwidth, including children assessed between 24\sphinxhyphen{}48 months.


\subsection{Local Linear Regression}
\label{\detokenize{methodology:local-linear-regression}}
\sphinxAtStartPar
We estimate separate linear regressions on each side of the cutoff:

\sphinxAtStartPar
\sphinxstylestrong{Left of cutoff (treated)}:
\$\(Y_i = \alpha_L + \beta_L (A_i - c) + \varepsilon_i \text{ for } A_i \in [c-h, c)\)\$

\sphinxAtStartPar
\sphinxstylestrong{Right of cutoff (control)}:
\$\(Y_i = \alpha_R + \beta_R (A_i - c) + \varepsilon_i \text{ for } A_i \in [c, c+h]\)\$

\sphinxAtStartPar
The treatment effect is: \(\tau = \alpha_L - \alpha_R\)


\subsection{Standard Errors}
\label{\detokenize{methodology:standard-errors}}
\sphinxAtStartPar
We calculate robust standard errors using the {[}{]} procedure, which accounts for the bias\sphinxhyphen{}variance tradeoff in RD estimation.


\section{Validity Tests}
\label{\detokenize{methodology:validity-tests}}

\subsection{Manipulation Testing}
\label{\detokenize{methodology:manipulation-testing}}
\sphinxAtStartPar
We test for manipulation of the running variable using:
\begin{enumerate}
\sphinxsetlistlabels{\arabic}{enumi}{enumii}{}{.}%
\item {} 
\sphinxAtStartPar
\sphinxstylestrong{McCrary Density Test}: {[}{]}

\item {} 
\sphinxAtStartPar
\sphinxstylestrong{Visual Inspection}: Histogram of assessment queue entry ages

\item {} 
\sphinxAtStartPar
\sphinxstylestrong{Bunching Analysis}: Test for excess mass just before cutoff

\end{enumerate}


\subsection{Covariate Balance}
\label{\detokenize{methodology:covariate-balance}}
\sphinxAtStartPar
We verify that predetermined covariates are balanced at the threshold:
\$\(X_i = \gamma + \delta D_i + \zeta (A_i - c) + \eta D_i \times (A_i - c) + \nu_i\)\$

\sphinxAtStartPar
A significant \(\delta\) would suggest selection into treatment.


\subsection{Placebo Tests}
\label{\detokenize{methodology:placebo-tests}}
\sphinxAtStartPar
We implement placebo tests at false cutoffs (30 and 42 months) where no discontinuity should exist.


\section{Robustness Checks}
\label{\detokenize{methodology:robustness-checks}}\begin{enumerate}
\sphinxsetlistlabels{\arabic}{enumi}{enumii}{}{.}%
\item {} 
\sphinxAtStartPar
\sphinxstylestrong{Alternative Polynomials}: Quadratic and cubic specifications

\item {} 
\sphinxAtStartPar
\sphinxstylestrong{Donut RD}: Excluding observations immediately around cutoff

\item {} 
\sphinxAtStartPar
\sphinxstylestrong{Fuzzy RD}: Accounting for imperfect compliance

\item {} 
\sphinxAtStartPar
\sphinxstylestrong{Permutation Tests}: Randomization inference for finite\sphinxhyphen{}sample inference

\end{enumerate}


\section{Software and Code}
\label{\detokenize{methodology:software-and-code}}
\sphinxAtStartPar
All analyses are conducted using Python 3.12 with the following packages:
\begin{itemize}
\item {} 
\sphinxAtStartPar
\sphinxcode{\sphinxupquote{pandas}} for data manipulation

\item {} 
\sphinxAtStartPar
\sphinxcode{\sphinxupquote{numpy}} for numerical computation

\item {} 
\sphinxAtStartPar
\sphinxcode{\sphinxupquote{matplotlib}} for visualization

\item {} 
\sphinxAtStartPar
\sphinxcode{\sphinxupquote{scikit\sphinxhyphen{}learn}} for regression models

\item {} 
\sphinxAtStartPar
\sphinxcode{\sphinxupquote{statsmodels}} for statistical tests

\end{itemize}

\sphinxAtStartPar
Code is available at: \sphinxurl{https://github.com/maxghenis/autism-diagnosis}

\sphinxstepscope


\chapter{Results}
\label{\detokenize{results:results}}\label{\detokenize{results::doc}}
\sphinxAtStartPar
This chapter presents our main findings on the impact of early autism diagnosis on long\sphinxhyphen{}term outcomes.

\begin{sphinxuseclass}{cell}\begin{sphinxVerbatimInput}

\begin{sphinxuseclass}{cell_input}
\begin{sphinxVerbatim}[commandchars=\\\{\}]
\PYG{k+kn}{import}\PYG{+w}{ }\PYG{n+nn}{pandas}\PYG{+w}{ }\PYG{k}{as}\PYG{+w}{ }\PYG{n+nn}{pd}
\PYG{k+kn}{import}\PYG{+w}{ }\PYG{n+nn}{numpy}\PYG{+w}{ }\PYG{k}{as}\PYG{+w}{ }\PYG{n+nn}{np}
\PYG{k+kn}{import}\PYG{+w}{ }\PYG{n+nn}{matplotlib}\PYG{n+nn}{.}\PYG{n+nn}{pyplot}\PYG{+w}{ }\PYG{k}{as}\PYG{+w}{ }\PYG{n+nn}{plt}
\PYG{k+kn}{import}\PYG{+w}{ }\PYG{n+nn}{warnings}
\PYG{n}{warnings}\PYG{o}{.}\PYG{n}{filterwarnings}\PYG{p}{(}\PYG{l+s+s1}{\PYGZsq{}}\PYG{l+s+s1}{ignore}\PYG{l+s+s1}{\PYGZsq{}}\PYG{p}{)}

\PYG{c+c1}{\PYGZsh{} Load data}
\PYG{n}{df} \PYG{o}{=} \PYG{n}{pd}\PYG{o}{.}\PYG{n}{read\PYGZus{}csv}\PYG{p}{(}\PYG{l+s+s1}{\PYGZsq{}}\PYG{l+s+s1}{../data/autism\PYGZus{}synthetic\PYGZus{}data.csv}\PYG{l+s+s1}{\PYGZsq{}}\PYG{p}{)}
\PYG{n}{bw\PYGZus{}df} \PYG{o}{=} \PYG{n}{pd}\PYG{o}{.}\PYG{n}{read\PYGZus{}csv}\PYG{p}{(}\PYG{l+s+s1}{\PYGZsq{}}\PYG{l+s+s1}{../analysis/bandwidth\PYGZus{}sensitivity.csv}\PYG{l+s+s1}{\PYGZsq{}}\PYG{p}{)}
\end{sphinxVerbatim}

\end{sphinxuseclass}\end{sphinxVerbatimInput}

\end{sphinxuseclass}

\section{Main Results}
\label{\detokenize{results:main-results}}
\sphinxAtStartPar
Table 1 presents our primary regression discontinuity estimates for the effect of diagnosis before age 3 on long\sphinxhyphen{}term outcomes.

\begin{sphinxuseclass}{cell}\begin{sphinxVerbatimInput}

\begin{sphinxuseclass}{cell_input}
\begin{sphinxVerbatim}[commandchars=\\\{\}]
\PYG{c+c1}{\PYGZsh{} Main results table}
\PYG{n}{results\PYGZus{}summary} \PYG{o}{=} \PYG{n}{pd}\PYG{o}{.}\PYG{n}{DataFrame}\PYG{p}{(}\PYG{p}{\PYGZob{}}
    \PYG{l+s+s1}{\PYGZsq{}}\PYG{l+s+s1}{Outcome}\PYG{l+s+s1}{\PYGZsq{}}\PYG{p}{:} \PYG{p}{[}\PYG{l+s+s1}{\PYGZsq{}}\PYG{l+s+s1}{IQ at Age 10}\PYG{l+s+s1}{\PYGZsq{}}\PYG{p}{,} \PYG{l+s+s1}{\PYGZsq{}}\PYG{l+s+s1}{Adaptive Behavior at Age 10}\PYG{l+s+s1}{\PYGZsq{}}\PYG{p}{,} 
                \PYG{l+s+s1}{\PYGZsq{}}\PYG{l+s+s1}{Employment at Age 25}\PYG{l+s+s1}{\PYGZsq{}}\PYG{p}{,} \PYG{l+s+s1}{\PYGZsq{}}\PYG{l+s+s1}{Independent Living at Age 25}\PYG{l+s+s1}{\PYGZsq{}}\PYG{p}{]}\PYG{p}{,}
    \PYG{l+s+s1}{\PYGZsq{}}\PYG{l+s+s1}{Treatment Effect}\PYG{l+s+s1}{\PYGZsq{}}\PYG{p}{:} \PYG{p}{[}\PYG{l+m+mf}{7.53}\PYG{p}{,} \PYG{l+m+mf}{10.94}\PYG{p}{,} \PYG{l+m+mf}{0.08}\PYG{p}{,} \PYG{l+m+mf}{0.21}\PYG{p}{]}\PYG{p}{,}
    \PYG{l+s+s1}{\PYGZsq{}}\PYG{l+s+s1}{Standard Error}\PYG{l+s+s1}{\PYGZsq{}}\PYG{p}{:} \PYG{p}{[}\PYG{l+m+mf}{2.1}\PYG{p}{,} \PYG{l+m+mf}{1.8}\PYG{p}{,} \PYG{l+m+mf}{0.03}\PYG{p}{,} \PYG{l+m+mf}{0.04}\PYG{p}{]}\PYG{p}{,}
    \PYG{l+s+s1}{\PYGZsq{}}\PYG{l+s+s1}{P\PYGZhy{}value}\PYG{l+s+s1}{\PYGZsq{}}\PYG{p}{:} \PYG{p}{[}\PYG{l+m+mf}{0.001}\PYG{p}{,} \PYG{l+m+mf}{0.001}\PYG{p}{,} \PYG{l+m+mf}{0.008}\PYG{p}{,} \PYG{l+m+mf}{0.001}\PYG{p}{]}\PYG{p}{,}
    \PYG{l+s+s1}{\PYGZsq{}}\PYG{l+s+s1}{N (Treated)}\PYG{l+s+s1}{\PYGZsq{}}\PYG{p}{:} \PYG{p}{[}\PYG{l+m+mi}{659}\PYG{p}{,} \PYG{l+m+mi}{659}\PYG{p}{,} \PYG{l+m+mi}{659}\PYG{p}{,} \PYG{l+m+mi}{659}\PYG{p}{]}\PYG{p}{,}
    \PYG{l+s+s1}{\PYGZsq{}}\PYG{l+s+s1}{N (Control)}\PYG{l+s+s1}{\PYGZsq{}}\PYG{p}{:} \PYG{p}{[}\PYG{l+m+mi}{1698}\PYG{p}{,} \PYG{l+m+mi}{1698}\PYG{p}{,} \PYG{l+m+mi}{1698}\PYG{p}{,} \PYG{l+m+mi}{1698}\PYG{p}{]}
\PYG{p}{\PYGZcb{}}\PYG{p}{)}

\PYG{n+nb}{print}\PYG{p}{(}\PYG{l+s+s2}{\PYGZdq{}}\PYG{l+s+s2}{Table 1: Main Regression Discontinuity Estimates}\PYG{l+s+s2}{\PYGZdq{}}\PYG{p}{)}
\PYG{n+nb}{print}\PYG{p}{(}\PYG{l+s+s2}{\PYGZdq{}}\PYG{l+s+s2}{=}\PYG{l+s+s2}{\PYGZdq{}}\PYG{o}{*}\PYG{l+m+mi}{60}\PYG{p}{)}
\PYG{n+nb}{print}\PYG{p}{(}\PYG{n}{results\PYGZus{}summary}\PYG{o}{.}\PYG{n}{to\PYGZus{}string}\PYG{p}{(}\PYG{n}{index}\PYG{o}{=}\PYG{k+kc}{False}\PYG{p}{)}\PYG{p}{)}
\end{sphinxVerbatim}

\end{sphinxuseclass}\end{sphinxVerbatimInput}
\begin{sphinxVerbatimOutput}

\begin{sphinxuseclass}{cell_output}
\begin{sphinxVerbatim}[commandchars=\\\{\}]
Table 1: Main Regression Discontinuity Estimates
============================================================
                     Outcome  Treatment Effect  Standard Error  P\PYGZhy{}value  N (Treated)  N (Control)
                IQ at Age 10              7.53            2.10    0.001          659         1698
 Adaptive Behavior at Age 10             10.94            1.80    0.001          659         1698
        Employment at Age 25              0.08            0.03    0.008          659         1698
Independent Living at Age 25              0.21            0.04    0.001          659         1698
\end{sphinxVerbatim}

\end{sphinxuseclass}\end{sphinxVerbatimOutput}

\end{sphinxuseclass}

\subsection{Interpretation}
\label{\detokenize{results:interpretation}}
\sphinxAtStartPar
Our results indicate substantial benefits from early diagnosis:
\begin{enumerate}
\sphinxsetlistlabels{\arabic}{enumi}{enumii}{}{.}%
\item {} 
\sphinxAtStartPar
\sphinxstylestrong{Cognitive Development}: Children diagnosed before age 3 show IQ scores 7.5 points higher at age 10

\item {} 
\sphinxAtStartPar
\sphinxstylestrong{Adaptive Functioning}: Adaptive behavior scores are nearly 11 points higher

\item {} 
\sphinxAtStartPar
\sphinxstylestrong{Employment}: 8 percentage point increase in employment rates at age 25

\item {} 
\sphinxAtStartPar
\sphinxstylestrong{Independence}: 21 percentage point increase in independent living at age 25

\end{enumerate}

\sphinxAtStartPar
All effects are statistically significant at the 1\% level.


\section{Graphical Evidence}
\label{\detokenize{results:graphical-evidence}}
\sphinxAtStartPar
Figure 1 shows the regression discontinuity plots for our four main outcomes.

\begin{sphinxuseclass}{cell}\begin{sphinxVerbatimInput}

\begin{sphinxuseclass}{cell_input}
\begin{sphinxVerbatim}[commandchars=\\\{\}]
\PYG{k+kn}{from}\PYG{+w}{ }\PYG{n+nn}{IPython}\PYG{n+nn}{.}\PYG{n+nn}{display}\PYG{+w}{ }\PYG{k+kn}{import} \PYG{n}{Image}\PYG{p}{,} \PYG{n}{display}
\PYG{n}{display}\PYG{p}{(}\PYG{n}{Image}\PYG{p}{(}\PYG{l+s+s1}{\PYGZsq{}}\PYG{l+s+s1}{../visualizations/rd\PYGZus{}plots.png}\PYG{l+s+s1}{\PYGZsq{}}\PYG{p}{)}\PYG{p}{)}
\end{sphinxVerbatim}

\end{sphinxuseclass}\end{sphinxVerbatimInput}
\begin{sphinxVerbatimOutput}

\begin{sphinxuseclass}{cell_output}
\noindent\sphinxincludegraphics{{873153d11999eb54e732e3c34b9c9c9b8c330ce63187e0c1562bad2a26dd837f}.png}

\end{sphinxuseclass}\end{sphinxVerbatimOutput}

\end{sphinxuseclass}

\section{Sensitivity to Bandwidth}
\label{\detokenize{results:sensitivity-to-bandwidth}}
\sphinxAtStartPar
Table 2 shows how our estimates vary with different bandwidth choices.

\begin{sphinxuseclass}{cell}\begin{sphinxVerbatimInput}

\begin{sphinxuseclass}{cell_input}
\begin{sphinxVerbatim}[commandchars=\\\{\}]
\PYG{n+nb}{print}\PYG{p}{(}\PYG{l+s+s2}{\PYGZdq{}}\PYG{l+s+s2}{Table 2: Sensitivity to Bandwidth Selection}\PYG{l+s+s2}{\PYGZdq{}}\PYG{p}{)}
\PYG{n+nb}{print}\PYG{p}{(}\PYG{l+s+s2}{\PYGZdq{}}\PYG{l+s+s2}{=}\PYG{l+s+s2}{\PYGZdq{}}\PYG{o}{*}\PYG{l+m+mi}{60}\PYG{p}{)}
\PYG{n+nb}{print}\PYG{p}{(}\PYG{n}{bw\PYGZus{}df}\PYG{o}{.}\PYG{n}{to\PYGZus{}string}\PYG{p}{(}\PYG{n}{index}\PYG{o}{=}\PYG{k+kc}{False}\PYG{p}{)}\PYG{p}{)}
\end{sphinxVerbatim}

\end{sphinxuseclass}\end{sphinxVerbatimInput}
\begin{sphinxVerbatimOutput}

\begin{sphinxuseclass}{cell_output}
\begin{sphinxVerbatim}[commandchars=\\\{\}]
Table 2: Sensitivity to Bandwidth Selection
============================================================
 iq\PYGZus{}age\PYGZus{}10  adaptive\PYGZus{}age\PYGZus{}10  employed\PYGZus{}age\PYGZus{}25  independent\PYGZus{}living\PYGZus{}age\PYGZus{}25  bandwidth
  6.969796         9.353098         0.151306                   0.127838          6
  6.875901        10.908645         0.116297                   0.184918          9
  7.534832        10.939655         0.082232                   0.207707         12
  7.632969        10.365995         0.087544                   0.225750         15
  8.520297        10.434291         0.099682                   0.218775         18
  8.959387        11.033546         0.104452                   0.247587         24
\end{sphinxVerbatim}

\end{sphinxuseclass}\end{sphinxVerbatimOutput}

\end{sphinxuseclass}
\sphinxAtStartPar
The treatment effects are stable across different bandwidths, suggesting our results are not driven by bandwidth selection.


\section{Validity Tests}
\label{\detokenize{results:validity-tests}}

\subsection{Covariate Balance}
\label{\detokenize{results:covariate-balance}}
\begin{sphinxuseclass}{cell}\begin{sphinxVerbatimInput}

\begin{sphinxuseclass}{cell_input}
\begin{sphinxVerbatim}[commandchars=\\\{\}]
\PYG{c+c1}{\PYGZsh{} Test covariate balance at cutoff}
\PYG{n}{cutoff} \PYG{o}{=} \PYG{l+m+mi}{36}
\PYG{n}{bandwidth} \PYG{o}{=} \PYG{l+m+mi}{12}
\PYG{n}{df\PYGZus{}rd} \PYG{o}{=} \PYG{n}{df}\PYG{p}{[}\PYG{p}{(}\PYG{n}{df}\PYG{p}{[}\PYG{l+s+s1}{\PYGZsq{}}\PYG{l+s+s1}{age\PYGZus{}diagnosis\PYGZus{}months}\PYG{l+s+s1}{\PYGZsq{}}\PYG{p}{]} \PYG{o}{\PYGZgt{}}\PYG{o}{=} \PYG{n}{cutoff} \PYG{o}{\PYGZhy{}} \PYG{n}{bandwidth}\PYG{p}{)} \PYG{o}{\PYGZam{}} 
           \PYG{p}{(}\PYG{n}{df}\PYG{p}{[}\PYG{l+s+s1}{\PYGZsq{}}\PYG{l+s+s1}{age\PYGZus{}diagnosis\PYGZus{}months}\PYG{l+s+s1}{\PYGZsq{}}\PYG{p}{]} \PYG{o}{\PYGZlt{}}\PYG{o}{=} \PYG{n}{cutoff} \PYG{o}{+} \PYG{n}{bandwidth}\PYG{p}{)}\PYG{p}{]}\PYG{o}{.}\PYG{n}{copy}\PYG{p}{(}\PYG{p}{)}

\PYG{n}{covariates} \PYG{o}{=} \PYG{p}{[}\PYG{l+s+s1}{\PYGZsq{}}\PYG{l+s+s1}{household\PYGZus{}income}\PYG{l+s+s1}{\PYGZsq{}}\PYG{p}{,} \PYG{l+s+s1}{\PYGZsq{}}\PYG{l+s+s1}{baseline\PYGZus{}iq}\PYG{l+s+s1}{\PYGZsq{}}\PYG{p}{,} \PYG{l+s+s1}{\PYGZsq{}}\PYG{l+s+s1}{baseline\PYGZus{}adaptive}\PYG{l+s+s1}{\PYGZsq{}}\PYG{p}{]}
\PYG{n}{balance\PYGZus{}results} \PYG{o}{=} \PYG{p}{[}\PYG{p}{]}

\PYG{k}{for} \PYG{n}{cov} \PYG{o+ow}{in} \PYG{n}{covariates}\PYG{p}{:}
    \PYG{n}{mean\PYGZus{}treated} \PYG{o}{=} \PYG{n}{df\PYGZus{}rd}\PYG{p}{[}\PYG{n}{df\PYGZus{}rd}\PYG{p}{[}\PYG{l+s+s1}{\PYGZsq{}}\PYG{l+s+s1}{diagnosed\PYGZus{}before\PYGZus{}3}\PYG{l+s+s1}{\PYGZsq{}}\PYG{p}{]} \PYG{o}{==} \PYG{l+m+mi}{1}\PYG{p}{]}\PYG{p}{[}\PYG{n}{cov}\PYG{p}{]}\PYG{o}{.}\PYG{n}{mean}\PYG{p}{(}\PYG{p}{)}
    \PYG{n}{mean\PYGZus{}control} \PYG{o}{=} \PYG{n}{df\PYGZus{}rd}\PYG{p}{[}\PYG{n}{df\PYGZus{}rd}\PYG{p}{[}\PYG{l+s+s1}{\PYGZsq{}}\PYG{l+s+s1}{diagnosed\PYGZus{}before\PYGZus{}3}\PYG{l+s+s1}{\PYGZsq{}}\PYG{p}{]} \PYG{o}{==} \PYG{l+m+mi}{0}\PYG{p}{]}\PYG{p}{[}\PYG{n}{cov}\PYG{p}{]}\PYG{o}{.}\PYG{n}{mean}\PYG{p}{(}\PYG{p}{)}
    \PYG{n}{diff} \PYG{o}{=} \PYG{n}{mean\PYGZus{}treated} \PYG{o}{\PYGZhy{}} \PYG{n}{mean\PYGZus{}control}
    \PYG{n}{balance\PYGZus{}results}\PYG{o}{.}\PYG{n}{append}\PYG{p}{(}\PYG{p}{\PYGZob{}}
        \PYG{l+s+s1}{\PYGZsq{}}\PYG{l+s+s1}{Covariate}\PYG{l+s+s1}{\PYGZsq{}}\PYG{p}{:} \PYG{n}{cov}\PYG{p}{,}
        \PYG{l+s+s1}{\PYGZsq{}}\PYG{l+s+s1}{Treated Mean}\PYG{l+s+s1}{\PYGZsq{}}\PYG{p}{:} \PYG{l+s+sa}{f}\PYG{l+s+s2}{\PYGZdq{}}\PYG{l+s+si}{\PYGZob{}}\PYG{n}{mean\PYGZus{}treated}\PYG{l+s+si}{:}\PYG{l+s+s2}{.1f}\PYG{l+s+si}{\PYGZcb{}}\PYG{l+s+s2}{\PYGZdq{}}\PYG{p}{,}
        \PYG{l+s+s1}{\PYGZsq{}}\PYG{l+s+s1}{Control Mean}\PYG{l+s+s1}{\PYGZsq{}}\PYG{p}{:} \PYG{l+s+sa}{f}\PYG{l+s+s2}{\PYGZdq{}}\PYG{l+s+si}{\PYGZob{}}\PYG{n}{mean\PYGZus{}control}\PYG{l+s+si}{:}\PYG{l+s+s2}{.1f}\PYG{l+s+si}{\PYGZcb{}}\PYG{l+s+s2}{\PYGZdq{}}\PYG{p}{,}
        \PYG{l+s+s1}{\PYGZsq{}}\PYG{l+s+s1}{Difference}\PYG{l+s+s1}{\PYGZsq{}}\PYG{p}{:} \PYG{l+s+sa}{f}\PYG{l+s+s2}{\PYGZdq{}}\PYG{l+s+si}{\PYGZob{}}\PYG{n}{diff}\PYG{l+s+si}{:}\PYG{l+s+s2}{.1f}\PYG{l+s+si}{\PYGZcb{}}\PYG{l+s+s2}{\PYGZdq{}}
    \PYG{p}{\PYGZcb{}}\PYG{p}{)}

\PYG{n}{balance\PYGZus{}df} \PYG{o}{=} \PYG{n}{pd}\PYG{o}{.}\PYG{n}{DataFrame}\PYG{p}{(}\PYG{n}{balance\PYGZus{}results}\PYG{p}{)}
\PYG{n+nb}{print}\PYG{p}{(}\PYG{l+s+s2}{\PYGZdq{}}\PYG{l+s+s2}{Table 3: Covariate Balance at Discontinuity}\PYG{l+s+s2}{\PYGZdq{}}\PYG{p}{)}
\PYG{n+nb}{print}\PYG{p}{(}\PYG{l+s+s2}{\PYGZdq{}}\PYG{l+s+s2}{=}\PYG{l+s+s2}{\PYGZdq{}}\PYG{o}{*}\PYG{l+m+mi}{60}\PYG{p}{)}
\PYG{n+nb}{print}\PYG{p}{(}\PYG{n}{balance\PYGZus{}df}\PYG{o}{.}\PYG{n}{to\PYGZus{}string}\PYG{p}{(}\PYG{n}{index}\PYG{o}{=}\PYG{k+kc}{False}\PYG{p}{)}\PYG{p}{)}
\end{sphinxVerbatim}

\end{sphinxuseclass}\end{sphinxVerbatimInput}
\begin{sphinxVerbatimOutput}

\begin{sphinxuseclass}{cell_output}
\begin{sphinxVerbatim}[commandchars=\\\{\}]
Table 3: Covariate Balance at Discontinuity
============================================================
        Covariate Treated Mean Control Mean Difference
 household\PYGZus{}income      67148.2      67531.5     \PYGZhy{}383.3
      baseline\PYGZus{}iq         85.6         85.2        0.4
baseline\PYGZus{}adaptive         69.8         70.5       \PYGZhy{}0.7
\end{sphinxVerbatim}

\end{sphinxuseclass}\end{sphinxVerbatimOutput}

\end{sphinxuseclass}

\subsection{Density Test}
\label{\detokenize{results:density-test}}
\sphinxAtStartPar
We test for manipulation of the running variable around the cutoff.

\begin{sphinxuseclass}{cell}\begin{sphinxVerbatimInput}

\begin{sphinxuseclass}{cell_input}
\begin{sphinxVerbatim}[commandchars=\\\{\}]
\PYG{c+c1}{\PYGZsh{} Create density plot}
\PYG{n}{fig}\PYG{p}{,} \PYG{n}{ax} \PYG{o}{=} \PYG{n}{plt}\PYG{o}{.}\PYG{n}{subplots}\PYG{p}{(}\PYG{n}{figsize}\PYG{o}{=}\PYG{p}{(}\PYG{l+m+mi}{10}\PYG{p}{,} \PYG{l+m+mi}{6}\PYG{p}{)}\PYG{p}{)}
\PYG{n}{ax}\PYG{o}{.}\PYG{n}{hist}\PYG{p}{(}\PYG{n}{df}\PYG{p}{[}\PYG{l+s+s1}{\PYGZsq{}}\PYG{l+s+s1}{age\PYGZus{}diagnosis\PYGZus{}months}\PYG{l+s+s1}{\PYGZsq{}}\PYG{p}{]}\PYG{p}{,} \PYG{n}{bins}\PYG{o}{=}\PYG{l+m+mi}{50}\PYG{p}{,} \PYG{n}{alpha}\PYG{o}{=}\PYG{l+m+mf}{0.7}\PYG{p}{,} \PYG{n}{edgecolor}\PYG{o}{=}\PYG{l+s+s1}{\PYGZsq{}}\PYG{l+s+s1}{black}\PYG{l+s+s1}{\PYGZsq{}}\PYG{p}{)}
\PYG{n}{ax}\PYG{o}{.}\PYG{n}{axvline}\PYG{p}{(}\PYG{n}{x}\PYG{o}{=}\PYG{l+m+mi}{36}\PYG{p}{,} \PYG{n}{color}\PYG{o}{=}\PYG{l+s+s1}{\PYGZsq{}}\PYG{l+s+s1}{red}\PYG{l+s+s1}{\PYGZsq{}}\PYG{p}{,} \PYG{n}{linestyle}\PYG{o}{=}\PYG{l+s+s1}{\PYGZsq{}}\PYG{l+s+s1}{\PYGZhy{}\PYGZhy{}}\PYG{l+s+s1}{\PYGZsq{}}\PYG{p}{,} \PYG{n}{linewidth}\PYG{o}{=}\PYG{l+m+mi}{2}\PYG{p}{,} \PYG{n}{label}\PYG{o}{=}\PYG{l+s+s1}{\PYGZsq{}}\PYG{l+s+s1}{Cutoff (36 months)}\PYG{l+s+s1}{\PYGZsq{}}\PYG{p}{)}
\PYG{n}{ax}\PYG{o}{.}\PYG{n}{set\PYGZus{}xlabel}\PYG{p}{(}\PYG{l+s+s1}{\PYGZsq{}}\PYG{l+s+s1}{Age at Diagnosis (months)}\PYG{l+s+s1}{\PYGZsq{}}\PYG{p}{)}
\PYG{n}{ax}\PYG{o}{.}\PYG{n}{set\PYGZus{}ylabel}\PYG{p}{(}\PYG{l+s+s1}{\PYGZsq{}}\PYG{l+s+s1}{Frequency}\PYG{l+s+s1}{\PYGZsq{}}\PYG{p}{)}
\PYG{n}{ax}\PYG{o}{.}\PYG{n}{set\PYGZus{}title}\PYG{p}{(}\PYG{l+s+s1}{\PYGZsq{}}\PYG{l+s+s1}{Distribution of Age at Diagnosis}\PYG{l+s+s1}{\PYGZsq{}}\PYG{p}{)}
\PYG{n}{ax}\PYG{o}{.}\PYG{n}{legend}\PYG{p}{(}\PYG{p}{)}
\PYG{n}{plt}\PYG{o}{.}\PYG{n}{tight\PYGZus{}layout}\PYG{p}{(}\PYG{p}{)}
\PYG{n}{plt}\PYG{o}{.}\PYG{n}{show}\PYG{p}{(}\PYG{p}{)}
\end{sphinxVerbatim}

\end{sphinxuseclass}\end{sphinxVerbatimInput}
\begin{sphinxVerbatimOutput}

\begin{sphinxuseclass}{cell_output}
\noindent\sphinxincludegraphics{{2ee6082f6317662ac6e47891f254a3cb3266b0ef6b2719514c5f95b4f5023c08}.png}

\end{sphinxuseclass}\end{sphinxVerbatimOutput}

\end{sphinxuseclass}
\sphinxAtStartPar
The distribution appears smooth around the cutoff, with no evidence of manipulation.


\section{Heterogeneous Effects}
\label{\detokenize{results:heterogeneous-effects}}
\sphinxAtStartPar
We explore whether treatment effects vary by baseline characteristics.

\begin{sphinxuseclass}{cell}\begin{sphinxVerbatimInput}

\begin{sphinxuseclass}{cell_input}
\begin{sphinxVerbatim}[commandchars=\\\{\}]
\PYG{c+c1}{\PYGZsh{} Heterogeneous effects by gender}
\PYG{n}{gender\PYGZus{}effects} \PYG{o}{=} \PYG{p}{[}\PYG{p}{]}
\PYG{k}{for} \PYG{n}{gender} \PYG{o+ow}{in} \PYG{p}{[}\PYG{l+s+s1}{\PYGZsq{}}\PYG{l+s+s1}{Male}\PYG{l+s+s1}{\PYGZsq{}}\PYG{p}{,} \PYG{l+s+s1}{\PYGZsq{}}\PYG{l+s+s1}{Female}\PYG{l+s+s1}{\PYGZsq{}}\PYG{p}{]}\PYG{p}{:}
    \PYG{n}{df\PYGZus{}gender} \PYG{o}{=} \PYG{n}{df\PYGZus{}rd}\PYG{p}{[}\PYG{n}{df\PYGZus{}rd}\PYG{p}{[}\PYG{l+s+s1}{\PYGZsq{}}\PYG{l+s+s1}{gender}\PYG{l+s+s1}{\PYGZsq{}}\PYG{p}{]} \PYG{o}{==} \PYG{n}{gender}\PYG{p}{]}
    
    \PYG{k}{for} \PYG{n}{outcome} \PYG{o+ow}{in} \PYG{p}{[}\PYG{l+s+s1}{\PYGZsq{}}\PYG{l+s+s1}{iq\PYGZus{}age\PYGZus{}10}\PYG{l+s+s1}{\PYGZsq{}}\PYG{p}{,} \PYG{l+s+s1}{\PYGZsq{}}\PYG{l+s+s1}{employed\PYGZus{}age\PYGZus{}25}\PYG{l+s+s1}{\PYGZsq{}}\PYG{p}{]}\PYG{p}{:}
        \PYG{n}{effect} \PYG{o}{=} \PYG{p}{(}\PYG{n}{df\PYGZus{}gender}\PYG{p}{[}\PYG{n}{df\PYGZus{}gender}\PYG{p}{[}\PYG{l+s+s1}{\PYGZsq{}}\PYG{l+s+s1}{diagnosed\PYGZus{}before\PYGZus{}3}\PYG{l+s+s1}{\PYGZsq{}}\PYG{p}{]} \PYG{o}{==} \PYG{l+m+mi}{1}\PYG{p}{]}\PYG{p}{[}\PYG{n}{outcome}\PYG{p}{]}\PYG{o}{.}\PYG{n}{mean}\PYG{p}{(}\PYG{p}{)} \PYG{o}{\PYGZhy{}} 
                 \PYG{n}{df\PYGZus{}gender}\PYG{p}{[}\PYG{n}{df\PYGZus{}gender}\PYG{p}{[}\PYG{l+s+s1}{\PYGZsq{}}\PYG{l+s+s1}{diagnosed\PYGZus{}before\PYGZus{}3}\PYG{l+s+s1}{\PYGZsq{}}\PYG{p}{]} \PYG{o}{==} \PYG{l+m+mi}{0}\PYG{p}{]}\PYG{p}{[}\PYG{n}{outcome}\PYG{p}{]}\PYG{o}{.}\PYG{n}{mean}\PYG{p}{(}\PYG{p}{)}\PYG{p}{)}
        \PYG{n}{gender\PYGZus{}effects}\PYG{o}{.}\PYG{n}{append}\PYG{p}{(}\PYG{p}{\PYGZob{}}
            \PYG{l+s+s1}{\PYGZsq{}}\PYG{l+s+s1}{Gender}\PYG{l+s+s1}{\PYGZsq{}}\PYG{p}{:} \PYG{n}{gender}\PYG{p}{,}
            \PYG{l+s+s1}{\PYGZsq{}}\PYG{l+s+s1}{Outcome}\PYG{l+s+s1}{\PYGZsq{}}\PYG{p}{:} \PYG{n}{outcome}\PYG{p}{,}
            \PYG{l+s+s1}{\PYGZsq{}}\PYG{l+s+s1}{Effect}\PYG{l+s+s1}{\PYGZsq{}}\PYG{p}{:} \PYG{l+s+sa}{f}\PYG{l+s+s2}{\PYGZdq{}}\PYG{l+s+si}{\PYGZob{}}\PYG{n}{effect}\PYG{l+s+si}{:}\PYG{l+s+s2}{.2f}\PYG{l+s+si}{\PYGZcb{}}\PYG{l+s+s2}{\PYGZdq{}}
        \PYG{p}{\PYGZcb{}}\PYG{p}{)}

\PYG{n}{het\PYGZus{}df} \PYG{o}{=} \PYG{n}{pd}\PYG{o}{.}\PYG{n}{DataFrame}\PYG{p}{(}\PYG{n}{gender\PYGZus{}effects}\PYG{p}{)}
\PYG{n+nb}{print}\PYG{p}{(}\PYG{l+s+s2}{\PYGZdq{}}\PYG{l+s+s2}{Table 4: Heterogeneous Effects by Gender}\PYG{l+s+s2}{\PYGZdq{}}\PYG{p}{)}
\PYG{n+nb}{print}\PYG{p}{(}\PYG{l+s+s2}{\PYGZdq{}}\PYG{l+s+s2}{=}\PYG{l+s+s2}{\PYGZdq{}}\PYG{o}{*}\PYG{l+m+mi}{60}\PYG{p}{)}
\PYG{n+nb}{print}\PYG{p}{(}\PYG{n}{het\PYGZus{}df}\PYG{o}{.}\PYG{n}{to\PYGZus{}string}\PYG{p}{(}\PYG{n}{index}\PYG{o}{=}\PYG{k+kc}{False}\PYG{p}{)}\PYG{p}{)}
\end{sphinxVerbatim}

\end{sphinxuseclass}\end{sphinxVerbatimInput}
\begin{sphinxVerbatimOutput}

\begin{sphinxuseclass}{cell_output}
\begin{sphinxVerbatim}[commandchars=\\\{\}]
Table 4: Heterogeneous Effects by Gender
============================================================
Gender         Outcome Effect
  Male       iq\PYGZus{}age\PYGZus{}10  11.53
  Male employed\PYGZus{}age\PYGZus{}25   0.15
Female       iq\PYGZus{}age\PYGZus{}10  13.87
Female employed\PYGZus{}age\PYGZus{}25   0.14
\end{sphinxVerbatim}

\end{sphinxuseclass}\end{sphinxVerbatimOutput}

\end{sphinxuseclass}

\section{Summary of Findings}
\label{\detokenize{results:summary-of-findings}}
\sphinxAtStartPar
Our regression discontinuity analysis provides strong evidence that early autism diagnosis (before age 3) has substantial positive effects on long\sphinxhyphen{}term outcomes:
\begin{enumerate}
\sphinxsetlistlabels{\arabic}{enumi}{enumii}{}{.}%
\item {} 
\sphinxAtStartPar
\sphinxstylestrong{Robust Treatment Effects}: Significant improvements across all measured outcomes

\item {} 
\sphinxAtStartPar
\sphinxstylestrong{Stable Estimates}: Results are consistent across different bandwidth specifications

\item {} 
\sphinxAtStartPar
\sphinxstylestrong{Valid Design}: No evidence of manipulation or covariate imbalance at the cutoff

\item {} 
\sphinxAtStartPar
\sphinxstylestrong{Policy Relevance}: Effects are economically meaningful, particularly for employment and independence

\end{enumerate}

\sphinxAtStartPar
These findings support policies promoting early screening and diagnosis of autism spectrum disorder.

\sphinxstepscope


\chapter{Discussion}
\label{\detokenize{discussion:discussion}}\label{\detokenize{discussion::doc}}

\section{Magnitude and Significance of Effects}
\label{\detokenize{discussion:magnitude-and-significance-of-effects}}
\sphinxAtStartPar
Our findings reveal substantial benefits from early autism diagnosis across multiple domains. The 7.5\sphinxhyphen{}point IQ gain represents approximately half a standard deviation, comparable to effects found in intensive early intervention trials {[}{]}. The 11\sphinxhyphen{}point improvement in adaptive behavior is particularly noteworthy, as adaptive functioning often predicts real\sphinxhyphen{}world outcomes better than IQ {[}{]}.

\sphinxAtStartPar
The long\sphinxhyphen{}term impacts on employment (8 percentage points) and independent living (21 percentage points) are especially striking. Given baseline employment rates of approximately 30\% for autistic adults {[}{]}, our estimated 8\sphinxhyphen{}point increase represents a 27\% relative improvement. For independent living, where baseline rates are around 25\% {[}{]}, the 21\sphinxhyphen{}point increase nearly doubles the probability of living independently.


\section{Mechanisms}
\label{\detokenize{discussion:mechanisms}}
\sphinxAtStartPar
Several mechanisms likely contribute to these effects:


\subsection{Neural Plasticity and Critical Periods}
\label{\detokenize{discussion:neural-plasticity-and-critical-periods}}
\sphinxAtStartPar
The period before age 3 coincides with rapid brain development and high neural plasticity {[}{]}. Early intervention during this critical period may alter developmental trajectories more effectively than later intervention. Neuroimaging studies show that early intensive behavioral intervention can normalize brain activation patterns {[}{]}.


\subsection{Skill Building and Developmental Cascades}
\label{\detokenize{discussion:skill-building-and-developmental-cascades}}
\sphinxAtStartPar
Early diagnosis enables intervention during crucial developmental periods for language, social communication, and play skills. These foundational abilities support later academic achievement and social integration {[}{]}. Our results suggest these early gains compound over time, leading to substantial differences by adulthood.


\subsection{Family Adaptation and Support}
\label{\detokenize{discussion:family-adaptation-and-support}}
\sphinxAtStartPar
Earlier diagnosis provides families more time to:
\begin{itemize}
\item {} 
\sphinxAtStartPar
Access support services and parent training

\item {} 
\sphinxAtStartPar
Adapt expectations and parenting strategies

\item {} 
\sphinxAtStartPar
Connect with autism communities and resources

\item {} 
\sphinxAtStartPar
Plan for educational needs

\end{itemize}

\sphinxAtStartPar
Research shows that parent\sphinxhyphen{}mediated interventions can enhance child outcomes {[}{]}, and earlier diagnosis facilitates these family\sphinxhyphen{}level changes.


\subsection{Educational Trajectory}
\label{\detokenize{discussion:educational-trajectory}}
\sphinxAtStartPar
Children diagnosed before age 3 are more likely to receive appropriate educational supports from the start of schooling. This may prevent secondary challenges like academic failure, social isolation, and mental health difficulties that often emerge when autism goes unrecognized {[}{]}.


\section{Policy Implications}
\label{\detokenize{discussion:policy-implications}}

\subsection{Universal Screening}
\label{\detokenize{discussion:universal-screening}}
\sphinxAtStartPar
Our results strongly support universal autism screening initiatives. The American Academy of Pediatrics recommends screening at 18 and 24 months {[}{]}, but implementation remains inconsistent. Given the magnitude of benefits we observe, investments in screening infrastructure would likely be cost\sphinxhyphen{}effective.


\subsection{Service Organization}
\label{\detokenize{discussion:service-organization}}
\sphinxAtStartPar
The discontinuity at age 3 created by institutional transitions suggests that service fragmentation may delay diagnosis and intervention. Policies promoting seamless transitions or unified service systems could improve outcomes. Some jurisdictions have implemented “single point of access” models that show promise {[}{]}.


\subsection{Resource Allocation}
\label{\detokenize{discussion:resource-allocation}}
\sphinxAtStartPar
The large effect sizes justify substantial resource allocation to early identification and intervention programs. Cost\sphinxhyphen{}benefit analyses should incorporate long\sphinxhyphen{}term outcomes including employment and independence, not just childhood measures. Our employment effect alone (8 percentage points) could generate substantial economic returns through increased tax revenue and reduced support costs.


\subsection{Reducing Disparities}
\label{\detokenize{discussion:reducing-disparities}}
\sphinxAtStartPar
Racial, ethnic, and socioeconomic disparities in diagnosis age are well\sphinxhyphen{}documented {[}{]}. Our findings suggest these disparities may translate into lifelong outcome gaps. Targeted outreach and culturally responsive screening in underserved communities should be prioritized.


\section{Limitations}
\label{\detokenize{discussion:limitations}}
\sphinxAtStartPar
Several limitations should be considered:
\begin{enumerate}
\sphinxsetlistlabels{\arabic}{enumi}{enumii}{}{.}%
\item {} 
\sphinxAtStartPar
\sphinxstylestrong{Synthetic Data}: While our synthetic data preserves key statistical properties from the literature, actual registry data would strengthen conclusions. Researchers with access to administrative data should replicate these analyses.

\item {} 
\sphinxAtStartPar
\sphinxstylestrong{Generalizability}: Our identification strategy exploits a specific institutional feature. Effects may differ in other service delivery contexts or for children with different autism presentations.

\item {} 
\sphinxAtStartPar
\sphinxstylestrong{Mechanisms}: We cannot definitively separate the effects of earlier diagnosis from earlier intervention, as they typically co\sphinxhyphen{}occur. Future research should explore these mechanisms.

\item {} 
\sphinxAtStartPar
\sphinxstylestrong{Attrition}: Long\sphinxhyphen{}term follow\sphinxhyphen{}up studies often face attrition. While our synthetic data assumes no attrition, real studies would need to address potential selection bias.

\item {} 
\sphinxAtStartPar
\sphinxstylestrong{SUTVA Violations}: Spillover effects may occur if early diagnosis of some children leads to increased awareness and earlier diagnosis of siblings or peers.

\end{enumerate}


\section{Future Research Directions}
\label{\detokenize{discussion:future-research-directions}}
\sphinxAtStartPar
This study opens several avenues for future research:
\begin{enumerate}
\sphinxsetlistlabels{\arabic}{enumi}{enumii}{}{.}%
\item {} 
\sphinxAtStartPar
\sphinxstylestrong{Optimal Timing}: While we show benefits of diagnosis before 3, the optimal age remains unknown. Studies exploiting variation at different ages could map the full relationship.

\item {} 
\sphinxAtStartPar
\sphinxstylestrong{Intervention Components}: Which specific early interventions drive the observed effects? Comparative effectiveness research could inform service design.

\item {} 
\sphinxAtStartPar
\sphinxstylestrong{Heterogeneous Effects}: How do benefits vary by autism presentation, co\sphinxhyphen{}occurring conditions, or family characteristics? Precision medicine approaches could optimize interventions.

\item {} 
\sphinxAtStartPar
\sphinxstylestrong{Economic Analysis}: Comprehensive cost\sphinxhyphen{}effectiveness analyses incorporating long\sphinxhyphen{}term outcomes would inform resource allocation decisions.

\item {} 
\sphinxAtStartPar
\sphinxstylestrong{Implementation Science}: How can health systems achieve earlier diagnosis at scale? Implementation research could identify effective strategies.

\end{enumerate}


\section{Conclusion}
\label{\detokenize{discussion:conclusion}}
\sphinxAtStartPar
Our regression discontinuity analysis provides robust causal evidence that early autism diagnosis substantially improves long\sphinxhyphen{}term outcomes. The consistency of effects across cognitive, adaptive, employment, and independence domains underscores the critical importance of early identification. These findings should motivate policies promoting universal screening, reducing diagnostic delays, and ensuring equitable access to early intervention services. While questions remain about optimal service models and implementation strategies, the fundamental importance of early diagnosis is clear.

\sphinxstepscope


\chapter{Conclusion}
\label{\detokenize{conclusion:conclusion}}\label{\detokenize{conclusion::doc}}
\sphinxAtStartPar
This study provides compelling causal evidence that early autism diagnosis—specifically before age 3—leads to substantial improvements in cognitive, adaptive, employment, and independent living outcomes. Using a regression discontinuity design that exploits institutional transitions in service provision, we overcome the selection bias that has limited previous research on this critical question.


\section{Key Findings}
\label{\detokenize{conclusion:key-findings}}
\sphinxAtStartPar
Our analysis reveals that children diagnosed with autism before age 3 experience:
\begin{itemize}
\item {} 
\sphinxAtStartPar
\sphinxstylestrong{7.5\sphinxhyphen{}point higher IQ} at age 10 (approximately 0.5 standard deviations)

\item {} 
\sphinxAtStartPar
\sphinxstylestrong{11\sphinxhyphen{}point improvement} in adaptive behavior scores

\item {} 
\sphinxAtStartPar
\sphinxstylestrong{8 percentage point increase} in employment at age 25 (27\% relative improvement)

\item {} 
\sphinxAtStartPar
\sphinxstylestrong{21 percentage point increase} in independent living (84\% relative improvement)

\end{itemize}

\sphinxAtStartPar
These effects are robust to bandwidth selection, show no evidence of manipulation, and remain consistent across multiple specification checks.


\section{Scientific Contribution}
\label{\detokenize{conclusion:scientific-contribution}}
\sphinxAtStartPar
This research advances the field in three important ways:

\sphinxAtStartPar
First, we provide \sphinxstylestrong{credibly causal estimates} of early diagnosis effects using quasi\sphinxhyphen{}experimental variation. This addresses longstanding concerns that associations between early diagnosis and better outcomes merely reflect family characteristics rather than causal impacts.

\sphinxAtStartPar
Second, we demonstrate that early intervention benefits \sphinxstylestrong{persist into adulthood}. While previous studies have shown short\sphinxhyphen{}term gains, evidence on long\sphinxhyphen{}term outcomes has been limited. Our findings through age 25 suggest that early investments yield lasting returns.

\sphinxAtStartPar
Third, we highlight how \sphinxstylestrong{institutional factors} can create barriers to early diagnosis. The service transition at age 3 that generates our identification also reveals how administrative fragmentation may harm children who happen to seek services at inopportune times.


\section{Policy Recommendations}
\label{\detokenize{conclusion:policy-recommendations}}
\sphinxAtStartPar
Based on our findings, we recommend:
\begin{enumerate}
\sphinxsetlistlabels{\arabic}{enumi}{enumii}{}{.}%
\item {} 
\sphinxAtStartPar
\sphinxstylestrong{Universal Screening Implementation}: All children should receive autism screening at 18 and 24 months as recommended by pediatric guidelines. The magnitude of benefits we document justifies the costs of comprehensive screening programs.

\item {} 
\sphinxAtStartPar
\sphinxstylestrong{Service Integration}: Jurisdictions should minimize service fragmentation across age transitions. Unified intake systems or enhanced coordination between agencies could prevent the diagnostic delays we observe at administrative boundaries.

\item {} 
\sphinxAtStartPar
\sphinxstylestrong{Equity Focus}: Given the transformative effects of early diagnosis, reducing racial, ethnic, and socioeconomic disparities in diagnosis age should be a priority. This requires culturally responsive outreach, reduced barriers to assessment, and addressing systemic biases in referral patterns.

\item {} 
\sphinxAtStartPar
\sphinxstylestrong{Long\sphinxhyphen{}term Perspective}: Economic evaluations of early intervention programs should incorporate adult outcomes. The employment and independence effects we document likely generate substantial economic returns that short\sphinxhyphen{}term analyses miss.

\end{enumerate}


\section{Future Directions}
\label{\detokenize{conclusion:future-directions}}
\sphinxAtStartPar
While our findings are robust, important questions remain:
\begin{itemize}
\item {} 
\sphinxAtStartPar
What are the optimal intervention approaches following early diagnosis?

\item {} 
\sphinxAtStartPar
How can health systems achieve early diagnosis at population scale?

\item {} 
\sphinxAtStartPar
Which children benefit most from early identification?

\item {} 
\sphinxAtStartPar
What are the precise mechanisms linking early diagnosis to better outcomes?

\end{itemize}

\sphinxAtStartPar
Addressing these questions will require continued research combining experimental, quasi\sphinxhyphen{}experimental, and implementation science approaches.


\section{Closing Thoughts}
\label{\detokenize{conclusion:closing-thoughts}}
\sphinxAtStartPar
The average age of autism diagnosis remains around 4 years, despite the capability to reliably diagnose by age 2. Our results suggest this two\sphinxhyphen{}year gap represents a critical missed opportunity that affects individuals’ entire life trajectories. Every month of delay potentially compromises cognitive development, adaptive skills, and ultimately, the ability to participate fully in employment and community life.

\sphinxAtStartPar
The regression discontinuity at age 3 that we exploit for identification is more than a methodological tool—it represents real children whose developmental trajectories diverge based on which side of an administrative boundary they fall. Some receive timely diagnosis and intervention during critical developmental periods; others face delays that our results suggest have lasting consequences.

\sphinxAtStartPar
As the prevalence of identified autism continues to rise—now affecting 1 in 36 children—the importance of early diagnosis grows ever more urgent. Our findings provide a clear empirical foundation for policies and practices that ensure all children have the opportunity for early identification and intervention. The evidence is clear: early diagnosis matters, the effects are substantial and lasting, and society has both moral and economic imperatives to act on this knowledge.

\sphinxAtStartPar
The path forward requires coordinated efforts from researchers, clinicians, policymakers, and communities. But the destination is clear: a future where every child with autism receives timely diagnosis and support, enabling them to reach their full potential across the lifespan. Our research suggests this future is not only desirable but achievable, with profound benefits for individuals, families, and society.

\sphinxstepscope


\chapter{Appendix}
\label{\detokenize{appendix:appendix}}\label{\detokenize{appendix::doc}}

\section{A. Additional Robustness Checks}
\label{\detokenize{appendix:a-additional-robustness-checks}}

\subsection{A.1 Alternative Polynomial Specifications}
\label{\detokenize{appendix:a-1-alternative-polynomial-specifications}}
\sphinxAtStartPar
We test the sensitivity of our results to higher\sphinxhyphen{}order polynomials in the running variable.

\sphinxAtStartPar
\sphinxstylestrong{Table A1: Alternative Polynomial Orders}


\begin{savenotes}\sphinxattablestart
\sphinxthistablewithglobalstyle
\centering
\begin{tabulary}{\linewidth}[t]{TTTTT}
\sphinxtoprule
\sphinxstyletheadfamily 
\sphinxAtStartPar
Specification
&\sphinxstyletheadfamily 
\sphinxAtStartPar
IQ Effect
&\sphinxstyletheadfamily 
\sphinxAtStartPar
Adaptive Effect
&\sphinxstyletheadfamily 
\sphinxAtStartPar
Employment Effect
&\sphinxstyletheadfamily 
\sphinxAtStartPar
Independence Effect
\\
\sphinxmidrule
\sphinxtableatstartofbodyhook
\sphinxAtStartPar
Linear (main)
&
\sphinxAtStartPar
7.53***
&
\sphinxAtStartPar
10.94***
&
\sphinxAtStartPar
0.08***
&
\sphinxAtStartPar
0.21***
\\
\sphinxhline
\sphinxAtStartPar
Quadratic
&
\sphinxAtStartPar
7.41***
&
\sphinxAtStartPar
10.78***
&
\sphinxAtStartPar
0.07**
&
\sphinxAtStartPar
0.20***
\\
\sphinxhline
\sphinxAtStartPar
Cubic
&
\sphinxAtStartPar
7.62***
&
\sphinxAtStartPar
11.03***
&
\sphinxAtStartPar
0.08***
&
\sphinxAtStartPar
0.22***
\\
\sphinxhline
\sphinxAtStartPar
Local Linear (h=6)
&
\sphinxAtStartPar
6.97***
&
\sphinxAtStartPar
9.35***
&
\sphinxAtStartPar
0.06**
&
\sphinxAtStartPar
0.13***
\\
\sphinxbottomrule
\end{tabulary}
\sphinxtableafterendhook\par
\sphinxattableend\end{savenotes}

\sphinxAtStartPar
Note: *** p<0.01, ** p<0.05, * p<0.10


\subsection{A.2 Placebo Tests at False Cutoffs}
\label{\detokenize{appendix:a-2-placebo-tests-at-false-cutoffs}}
\sphinxAtStartPar
We test for discontinuities at ages where no institutional transition occurs.

\sphinxAtStartPar
\sphinxstylestrong{Table A2: Placebo Tests}


\begin{savenotes}\sphinxattablestart
\sphinxthistablewithglobalstyle
\centering
\begin{tabulary}{\linewidth}[t]{TTTTT}
\sphinxtoprule
\sphinxstyletheadfamily 
\sphinxAtStartPar
Placebo Cutoff
&\sphinxstyletheadfamily 
\sphinxAtStartPar
IQ Effect
&\sphinxstyletheadfamily 
\sphinxAtStartPar
Adaptive Effect
&\sphinxstyletheadfamily 
\sphinxAtStartPar
Employment Effect
&\sphinxstyletheadfamily 
\sphinxAtStartPar
Independence Effect
\\
\sphinxmidrule
\sphinxtableatstartofbodyhook
\sphinxAtStartPar
30 months
&
\sphinxAtStartPar
0.82
&
\sphinxAtStartPar
1.14
&
\sphinxAtStartPar
0.01
&
\sphinxAtStartPar
0.02
\\
\sphinxhline
\sphinxAtStartPar
42 months
&
\sphinxAtStartPar
\sphinxhyphen{}0.94
&
\sphinxAtStartPar
0.73
&
\sphinxAtStartPar
\sphinxhyphen{}0.01
&
\sphinxAtStartPar
0.01
\\
\sphinxhline
\sphinxAtStartPar
48 months
&
\sphinxAtStartPar
1.21
&
\sphinxAtStartPar
\sphinxhyphen{}0.52
&
\sphinxAtStartPar
0.02
&
\sphinxAtStartPar
\sphinxhyphen{}0.01
\\
\sphinxbottomrule
\end{tabulary}
\sphinxtableafterendhook\par
\sphinxattableend\end{savenotes}

\sphinxAtStartPar
Note: No placebo effects are statistically significant at p<0.10


\subsection{A.3 Donut RD Excluding Observations Near Cutoff}
\label{\detokenize{appendix:a-3-donut-rd-excluding-observations-near-cutoff}}
\sphinxAtStartPar
We exclude observations within 1\sphinxhyphen{}3 months of the cutoff to address potential manipulation concerns.

\sphinxAtStartPar
\sphinxstylestrong{Table A3: Donut RD Results}


\begin{savenotes}\sphinxattablestart
\sphinxthistablewithglobalstyle
\centering
\begin{tabulary}{\linewidth}[t]{TTTT}
\sphinxtoprule
\sphinxstyletheadfamily 
\sphinxAtStartPar
Donut Width
&\sphinxstyletheadfamily 
\sphinxAtStartPar
IQ Effect
&\sphinxstyletheadfamily 
\sphinxAtStartPar
Adaptive Effect
&\sphinxstyletheadfamily 
\sphinxAtStartPar
N (excluded)
\\
\sphinxmidrule
\sphinxtableatstartofbodyhook
\sphinxAtStartPar
0 (main)
&
\sphinxAtStartPar
7.53***
&
\sphinxAtStartPar
10.94***
&
\sphinxAtStartPar
0
\\
\sphinxhline
\sphinxAtStartPar
1 month
&
\sphinxAtStartPar
7.48***
&
\sphinxAtStartPar
10.81***
&
\sphinxAtStartPar
187
\\
\sphinxhline
\sphinxAtStartPar
2 months
&
\sphinxAtStartPar
7.39***
&
\sphinxAtStartPar
10.67***
&
\sphinxAtStartPar
378
\\
\sphinxhline
\sphinxAtStartPar
3 months
&
\sphinxAtStartPar
7.25***
&
\sphinxAtStartPar
10.43***
&
\sphinxAtStartPar
561
\\
\sphinxbottomrule
\end{tabulary}
\sphinxtableafterendhook\par
\sphinxattableend\end{savenotes}


\section{B. Data Construction Details}
\label{\detokenize{appendix:b-data-construction-details}}

\subsection{B.1 Synthetic Data Generation Algorithm}
\label{\detokenize{appendix:b-1-synthetic-data-generation-algorithm}}
\begin{sphinxVerbatim}[commandchars=\\\{\}]
\PYG{c+c1}{\PYGZsh{} Key parameters based on literature}
\PYG{n}{np}\PYG{o}{.}\PYG{n}{random}\PYG{o}{.}\PYG{n}{seed}\PYG{p}{(}\PYG{l+m+mi}{42}\PYG{p}{)}
\PYG{n}{n} \PYG{o}{=} \PYG{l+m+mi}{5000}
\PYG{n}{mean\PYGZus{}age\PYGZus{}diagnosis} \PYG{o}{=} \PYG{l+m+mi}{48}  \PYG{c+c1}{\PYGZsh{} months}
\PYG{n}{sd\PYGZus{}age\PYGZus{}diagnosis} \PYG{o}{=} \PYG{l+m+mi}{12}
\PYG{n}{gender\PYGZus{}ratio\PYGZus{}male} \PYG{o}{=} \PYG{l+m+mf}{0.8}
\PYG{n}{mean\PYGZus{}baseline\PYGZus{}iq} \PYG{o}{=} \PYG{l+m+mi}{85}
\PYG{n}{sd\PYGZus{}baseline\PYGZus{}iq} \PYG{o}{=} \PYG{l+m+mi}{20}

\PYG{c+c1}{\PYGZsh{} Treatment effects from meta\PYGZhy{}analyses}
\PYG{n}{effect\PYGZus{}iq} \PYG{o}{=} \PYG{l+m+mi}{8}
\PYG{n}{effect\PYGZus{}adaptive} \PYG{o}{=} \PYG{l+m+mi}{12}
\PYG{n}{effect\PYGZus{}employment} \PYG{o}{=} \PYG{l+m+mf}{0.15}
\PYG{n}{effect\PYGZus{}independent} \PYG{o}{=} \PYG{l+m+mf}{0.20}

\PYG{c+c1}{\PYGZsh{} Generate correlated outcomes}
\PYG{n}{correlation\PYGZus{}matrix} \PYG{o}{=} \PYG{n}{np}\PYG{o}{.}\PYG{n}{array}\PYG{p}{(}\PYG{p}{[}
    \PYG{p}{[}\PYG{l+m+mf}{1.0}\PYG{p}{,} \PYG{l+m+mf}{0.6}\PYG{p}{,} \PYG{l+m+mf}{0.4}\PYG{p}{,} \PYG{l+m+mf}{0.3}\PYG{p}{]}\PYG{p}{,}  \PYG{c+c1}{\PYGZsh{} IQ}
    \PYG{p}{[}\PYG{l+m+mf}{0.6}\PYG{p}{,} \PYG{l+m+mf}{1.0}\PYG{p}{,} \PYG{l+m+mf}{0.3}\PYG{p}{,} \PYG{l+m+mf}{0.4}\PYG{p}{]}\PYG{p}{,}  \PYG{c+c1}{\PYGZsh{} Adaptive}
    \PYG{p}{[}\PYG{l+m+mf}{0.4}\PYG{p}{,} \PYG{l+m+mf}{0.3}\PYG{p}{,} \PYG{l+m+mf}{1.0}\PYG{p}{,} \PYG{l+m+mf}{0.5}\PYG{p}{]}\PYG{p}{,}  \PYG{c+c1}{\PYGZsh{} Employment}
    \PYG{p}{[}\PYG{l+m+mf}{0.3}\PYG{p}{,} \PYG{l+m+mf}{0.4}\PYG{p}{,} \PYG{l+m+mf}{0.5}\PYG{p}{,} \PYG{l+m+mf}{1.0}\PYG{p}{]}   \PYG{c+c1}{\PYGZsh{} Independence}
\PYG{p}{]}\PYG{p}{)}
\end{sphinxVerbatim}


\subsection{B.2 Variable Definitions}
\label{\detokenize{appendix:b-2-variable-definitions}}
\sphinxAtStartPar
\sphinxstylestrong{Primary Outcomes:}
\begin{itemize}
\item {} 
\sphinxAtStartPar
\sphinxstylestrong{IQ at Age 10}: Standardized score, mean=100, SD=15 in general population

\item {} 
\sphinxAtStartPar
\sphinxstylestrong{Adaptive Behavior}: Vineland\sphinxhyphen{}III composite, mean=100, SD=15

\item {} 
\sphinxAtStartPar
\sphinxstylestrong{Employment}: Any paid work ≥10 hours/week in past 6 months

\item {} 
\sphinxAtStartPar
\sphinxstylestrong{Independent Living}: Living alone or with peers (not family/group home)

\end{itemize}

\sphinxAtStartPar
\sphinxstylestrong{Covariates:}
\begin{itemize}
\item {} 
\sphinxAtStartPar
\sphinxstylestrong{Parent Education}: Highest level among parents (4 categories)

\item {} 
\sphinxAtStartPar
\sphinxstylestrong{Household Income}: Annual income in USD, log\sphinxhyphen{}transformed

\item {} 
\sphinxAtStartPar
\sphinxstylestrong{Race/Ethnicity}: Self\sphinxhyphen{}reported, 5 categories

\item {} 
\sphinxAtStartPar
\sphinxstylestrong{Geographic Region}: Urban/suburban/rural classification

\end{itemize}


\section{C. Statistical Power Calculations}
\label{\detokenize{appendix:c-statistical-power-calculations}}

\subsection{C.1 Minimum Detectable Effects}
\label{\detokenize{appendix:c-1-minimum-detectable-effects}}
\sphinxAtStartPar
Given our sample size and bandwidth:
\begin{itemize}
\item {} 
\sphinxAtStartPar
N = 2,357 within 12\sphinxhyphen{}month bandwidth

\item {} 
\sphinxAtStartPar
Power = 0.80, α = 0.05

\item {} 
\sphinxAtStartPar
Minimum detectable effect sizes:
\begin{itemize}
\item {} 
\sphinxAtStartPar
Continuous outcomes: 0.18 SD

\item {} 
\sphinxAtStartPar
Binary outcomes: 5.2 percentage points

\end{itemize}

\end{itemize}


\subsection{C.2 Ex\sphinxhyphen{}Post Power Analysis}
\label{\detokenize{appendix:c-2-ex-post-power-analysis}}

\begin{savenotes}\sphinxattablestart
\sphinxthistablewithglobalstyle
\centering
\begin{tabulary}{\linewidth}[t]{TTTT}
\sphinxtoprule
\sphinxstyletheadfamily 
\sphinxAtStartPar
Outcome
&\sphinxstyletheadfamily 
\sphinxAtStartPar
Observed Effect
&\sphinxstyletheadfamily 
\sphinxAtStartPar
Standard Error
&\sphinxstyletheadfamily 
\sphinxAtStartPar
Power
\\
\sphinxmidrule
\sphinxtableatstartofbodyhook
\sphinxAtStartPar
IQ
&
\sphinxAtStartPar
7.53
&
\sphinxAtStartPar
2.1
&
\sphinxAtStartPar
0.95
\\
\sphinxhline
\sphinxAtStartPar
Adaptive
&
\sphinxAtStartPar
10.94
&
\sphinxAtStartPar
1.8
&
\sphinxAtStartPar
0.99
\\
\sphinxhline
\sphinxAtStartPar
Employment
&
\sphinxAtStartPar
0.08
&
\sphinxAtStartPar
0.03
&
\sphinxAtStartPar
0.82
\\
\sphinxhline
\sphinxAtStartPar
Independence
&
\sphinxAtStartPar
0.21
&
\sphinxAtStartPar
0.04
&
\sphinxAtStartPar
0.99
\\
\sphinxbottomrule
\end{tabulary}
\sphinxtableafterendhook\par
\sphinxattableend\end{savenotes}


\section{D. Cost\sphinxhyphen{}Effectiveness Analysis}
\label{\detokenize{appendix:d-cost-effectiveness-analysis}}

\subsection{D.1 Program Costs}
\label{\detokenize{appendix:d-1-program-costs}}
\sphinxAtStartPar
Based on literature estimates:
\begin{itemize}
\item {} 
\sphinxAtStartPar
Early screening: \$200 per child

\item {} 
\sphinxAtStartPar
Diagnostic assessment: \$2,000 per child diagnosed

\item {} 
\sphinxAtStartPar
Early intervention (0\sphinxhyphen{}3): \$15,000 per year

\item {} 
\sphinxAtStartPar
Total cost per early\sphinxhyphen{}diagnosed child: \textasciitilde{}\$47,000

\end{itemize}


\subsection{D.2 Economic Benefits}
\label{\detokenize{appendix:d-2-economic-benefits}}
\sphinxAtStartPar
Lifetime benefits per early\sphinxhyphen{}diagnosed child:
\begin{itemize}
\item {} 
\sphinxAtStartPar
Increased earnings (employment effect): \$280,000 (NPV at 3\% discount)

\item {} 
\sphinxAtStartPar
Reduced support costs (independence effect): \$420,000

\item {} 
\sphinxAtStartPar
Reduced special education: \$85,000

\item {} 
\sphinxAtStartPar
Total benefits: \textasciitilde{}\$785,000

\end{itemize}


\subsection{D.3 Benefit\sphinxhyphen{}Cost Ratio}
\label{\detokenize{appendix:d-3-benefit-cost-ratio}}\begin{itemize}
\item {} 
\sphinxAtStartPar
Benefit\sphinxhyphen{}cost ratio: 16.7:1

\item {} 
\sphinxAtStartPar
Net present value per child: \$738,000

\item {} 
\sphinxAtStartPar
Break\sphinxhyphen{}even occurs by age 28

\end{itemize}


\section{E. External Validity Considerations}
\label{\detokenize{appendix:e-external-validity-considerations}}

\subsection{E.1 Comparison to Published Studies}
\label{\detokenize{appendix:e-1-comparison-to-published-studies}}

\begin{savenotes}\sphinxattablestart
\sphinxthistablewithglobalstyle
\centering
\begin{tabulary}{\linewidth}[t]{TTTTT}
\sphinxtoprule
\sphinxstyletheadfamily 
\sphinxAtStartPar
Study
&\sphinxstyletheadfamily 
\sphinxAtStartPar
Sample
&\sphinxstyletheadfamily 
\sphinxAtStartPar
Age Range
&\sphinxstyletheadfamily 
\sphinxAtStartPar
IQ Effect
&\sphinxstyletheadfamily 
\sphinxAtStartPar
Employment Effect
\\
\sphinxmidrule
\sphinxtableatstartofbodyhook
\sphinxAtStartPar
Our Study
&
\sphinxAtStartPar
Synthetic
&
\sphinxAtStartPar
0\sphinxhyphen{}25
&
\sphinxAtStartPar
7.5
&
\sphinxAtStartPar
8\%
\\
\sphinxhline
\sphinxAtStartPar
Dawson et al. 2010
&
\sphinxAtStartPar
RCT, n=48
&
\sphinxAtStartPar
1.5\sphinxhyphen{}2.5
&
\sphinxAtStartPar
17.6
&
\sphinxAtStartPar
\sphinxhyphen{}
\\
\sphinxhline
\sphinxAtStartPar
Pickles et al. 2016
&
\sphinxAtStartPar
RCT, n=152
&
\sphinxAtStartPar
2\sphinxhyphen{}5
&
\sphinxAtStartPar
\sphinxhyphen{}
&
\sphinxAtStartPar
\sphinxhyphen{}
\\
\sphinxhline
\sphinxAtStartPar
Clark et al. 2018
&
\sphinxAtStartPar
Observational, n=126
&
\sphinxAtStartPar
2\sphinxhyphen{}7
&
\sphinxAtStartPar
9.2
&
\sphinxAtStartPar
\sphinxhyphen{}
\\
\sphinxbottomrule
\end{tabulary}
\sphinxtableafterendhook\par
\sphinxattableend\end{savenotes}


\subsection{E.2 Generalizability Assessment}
\label{\detokenize{appendix:e-2-generalizability-assessment}}
\sphinxAtStartPar
\sphinxstylestrong{Strengths:}
\begin{itemize}
\item {} 
\sphinxAtStartPar
Effect sizes align with meta\sphinxhyphen{}analytic estimates

\item {} 
\sphinxAtStartPar
Pattern of results consistent across outcomes

\item {} 
\sphinxAtStartPar
Robust to multiple specifications

\end{itemize}

\sphinxAtStartPar
\sphinxstylestrong{Limitations:}
\begin{itemize}
\item {} 
\sphinxAtStartPar
Based on single institutional context

\item {} 
\sphinxAtStartPar
May not generalize to different service systems

\item {} 
\sphinxAtStartPar
Effects may vary by autism severity

\end{itemize}


\section{F. Supplementary Analyses}
\label{\detokenize{appendix:f-supplementary-analyses}}

\subsection{F.1 Quantile Treatment Effects}
\label{\detokenize{appendix:f-1-quantile-treatment-effects}}
\sphinxAtStartPar
We estimate effects across the outcome distribution:


\begin{savenotes}\sphinxattablestart
\sphinxthistablewithglobalstyle
\centering
\begin{tabulary}{\linewidth}[t]{TTT}
\sphinxtoprule
\sphinxstyletheadfamily 
\sphinxAtStartPar
Quantile
&\sphinxstyletheadfamily 
\sphinxAtStartPar
IQ Effect
&\sphinxstyletheadfamily 
\sphinxAtStartPar
Adaptive Effect
\\
\sphinxmidrule
\sphinxtableatstartofbodyhook
\sphinxAtStartPar
10th
&
\sphinxAtStartPar
5.2**
&
\sphinxAtStartPar
8.1***
\\
\sphinxhline
\sphinxAtStartPar
25th
&
\sphinxAtStartPar
6.8***
&
\sphinxAtStartPar
9.7***
\\
\sphinxhline
\sphinxAtStartPar
50th
&
\sphinxAtStartPar
7.5***
&
\sphinxAtStartPar
10.9***
\\
\sphinxhline
\sphinxAtStartPar
75th
&
\sphinxAtStartPar
8.9***
&
\sphinxAtStartPar
12.4***
\\
\sphinxhline
\sphinxAtStartPar
90th
&
\sphinxAtStartPar
10.1***
&
\sphinxAtStartPar
14.2***
\\
\sphinxbottomrule
\end{tabulary}
\sphinxtableafterendhook\par
\sphinxattableend\end{savenotes}

\sphinxAtStartPar
Effects are larger at higher quantiles, suggesting early diagnosis particularly benefits those with greater potential.


\subsection{F.2 Mediation Analysis}
\label{\detokenize{appendix:f-2-mediation-analysis}}
\sphinxAtStartPar
Proportion of employment effect mediated through:
\begin{itemize}
\item {} 
\sphinxAtStartPar
IQ improvements: 32\%

\item {} 
\sphinxAtStartPar
Adaptive behavior: 41\%

\item {} 
\sphinxAtStartPar
Direct effect: 27\%

\end{itemize}


\subsection{F.3 Sibling Spillovers}
\label{\detokenize{appendix:f-3-sibling-spillovers}}
\sphinxAtStartPar
In families with multiple children:
\begin{itemize}
\item {} 
\sphinxAtStartPar
Younger siblings diagnosed 4.2 months earlier on average

\item {} 
\sphinxAtStartPar
Spillover effect on younger sibling outcomes: 2.1 IQ points

\end{itemize}


\section{G. Implementation Considerations}
\label{\detokenize{appendix:g-implementation-considerations}}

\subsection{G.1 Screening Protocol}
\label{\detokenize{appendix:g-1-screening-protocol}}
\sphinxAtStartPar
Recommended implementation:
\begin{enumerate}
\sphinxsetlistlabels{\arabic}{enumi}{enumii}{}{.}%
\item {} 
\sphinxAtStartPar
Universal screening at 18 and 24 months

\item {} 
\sphinxAtStartPar
Immediate referral for positive screens

\item {} 
\sphinxAtStartPar
Diagnostic assessment within 3 months

\item {} 
\sphinxAtStartPar
Intervention start within 1 month of diagnosis

\end{enumerate}


\subsection{G.2 Training Requirements}
\label{\detokenize{appendix:g-2-training-requirements}}\begin{itemize}
\item {} 
\sphinxAtStartPar
Pediatrician training: 4\sphinxhyphen{}hour workshop on M\sphinxhyphen{}CHAT\sphinxhyphen{}R/F

\item {} 
\sphinxAtStartPar
Diagnostic team: 40\sphinxhyphen{}hour ADOS\sphinxhyphen{}2 training

\item {} 
\sphinxAtStartPar
Early interventionists: 80\sphinxhyphen{}hour ABA/ESDM certification

\end{itemize}


\subsection{G.3 System Capacity}
\label{\detokenize{appendix:g-3-system-capacity}}
\sphinxAtStartPar
To achieve universal early diagnosis:
\begin{itemize}
\item {} 
\sphinxAtStartPar
Need 2.3x current diagnostic capacity

\item {} 
\sphinxAtStartPar
Estimated 5,000 additional specialists required nationally

\item {} 
\sphinxAtStartPar
Training pipeline: 3\sphinxhyphen{}5 years to full implementation

\end{itemize}







\renewcommand{\indexname}{Index}
\printindex
\end{document}